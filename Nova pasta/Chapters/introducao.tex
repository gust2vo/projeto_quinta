\chapter{Introdução}
\label{chap:intro}

% Este pode ser um parágrafo citado por alguém \cite{Barabasi2003-1} e \cite{barabasi2003linked}.
% Para ajustar veja o comentário do capítulo \ref{chap:fundteor}.

% As orientações do robô \cite{aperea-1}.
Este relatório consiste em apresentar um projeto prático sobre Análise e Projeto de Sistemas, nele ira conter toda a documentação sobre o nosso projeto que consiste na criação de um site portifolio. Buscamos entender todas as solicitações e necessidades do nosso cliente para realizar a criação deste site. Neste relatório vai conter todos os nossos diagramas, reuniões com o cliente e equipe, explicação técnica e demais itens que englobam a Análise e Projeto de Sistemas.

% fakdfjlsdjfldsjfldsj
% dfkhfdskfhkdjh


% Segundo \citeonline{barabasi2003linked}, ...

% 
% \loremipsum dolor sit amet, consectetur adipiscing elit. Sed do eiusmod tempor incididunt ut labore et dolore magna aliqua. Ut enim ad minim veniam, quis nostrud exercitation ullamco laboris nisi ut aliquip ex ea commodo consequat. Duis aute irure dolor in reprehenderit in voluptate velit esse cillum dolore eu fugiat nulla pariatur. Excepteur sint occaecat cupidatat non proident, sunt in culpa qui officia deserunt mollit anim id est laborum.
%--------- NEW SECTION ----------------------
\section{Objetivos}
\label{sec:obj}
Este projeto tem como objetivo desenvolver um site portfólio para um cliente real, aplicando os conhecimentos de Análise e Projeto de Sistemas. O trabalho abrange desde o levantamento de requisitos até a entrega do produto final, incluindo toda a documentação necessária, como diagramas, registros de reuniões, decisões técnicas e justificativas de projeto. Busca-se atender às necessidades específicas do cliente por meio de uma solução funcional, bem documentada e tecnicamente estruturada. 
\label{sec:obj}

\subsection{Objetivos Específicos}
\label{ssec:objesp}
Os objetivos específicos deste projeto são:
\begin{itemize}
      \item Levantar e documentar os requisitos do cliente por meio de reuniões e entrevistas;
      \item Elaborar os diagramas UML necessários, como casos de uso, atividades e classes;
      \item Definir e justificar as decisões técnicas adotadas no desenvolvimento do sistema;
      \item Desenvolver e validar o site portfólio com base nas necessidades identificadas.
  \end{itemize}

%--------- NEW SECTION ----------------------
\section{Justificativa}
\label{sec:justi}

A realização deste projeto justifica-se pela oportunidade de aplicar, na prática, os conhecimentos adquiridos na disciplina de Análise e Projeto de Sistemas. Ao desenvolver um site portfólio real para um cliente, os integrantes da equipe têm a chance de vivenciar o ciclo completo de desenvolvimento de software, desde o levantamento de requisitos até a entrega do produto final.

Além disso, o projeto proporciona o contato direto com o cliente, permitindo o desenvolvimento de habilidades interpessoais, como comunicação, escuta ativa e trabalho em equipe, fundamentais para o sucesso de qualquer projeto na área de tecnologia.

A documentação detalhada de todas as etapas — incluindo reuniões, diagramas, decisões técnicas e validações — contribui para consolidar o aprendizado e formar uma base sólida para futuros projetos acadêmicos ou profissionais.

São apresentados o diagrama de casos de uso, o diagrama de sequência e o diagrama de classes, que evidenciam as funcionalidades, a arquitetura e as interações do sistema desenvolvido \cite{Larman2007}.

Por fim, a entrega de um produto funcional que atenda às necessidades reais de um cliente agrega valor social e acadêmico, evidenciando a relevância prática da disciplina para o mercado de trabalho \cite{Pressman2016, Sommerville2011}.




%--------- NEW SECTION ----------------------
\section{Organização do documento}
\label{section:organizacao}

Este documento apresenta $5$ capítulos e está estruturado da seguinte forma:

\begin{itemize}

\item \textbf{Capítulo \ref{chap:intro} - Introdução}: Contextualiza o âmbito no qual a pesquisa proposta está inserida. Apresenta, portanto, a definição do problema, objetivos e justificativas da pesquisa e como este \thetypeworkthree está estruturado;

\item \textbf{Capítulo \ref{chap:fundteor} - Fundamentação Teórica}: Descreve os principais conceitos e tecnologias utilizadas na construção do site portfólio, incluindo HTML, CSS, JavaScript e a plataforma de hospedagem GitHub Pages. Cada tecnologia é abordada em termos de suas funcionalidades e contribuições para a estrutura, estilo, interatividade e publicação do projeto;

\item \textbf{Capítulo \ref{chap:metod} - Materiais e Métodos}: Apresenta os procedimentos metodológicos adotados para o desenvolvimento do projeto. Explica a abordagem híbrida baseada nos modelos Waterfall e Scrum, além das fases de planejamento (Gate A) e design/validação (Gate B), contemplando desde o levantamento de requisitos até a prototipação inicial;

\item \textbf{Capítulo \ref{chap:result} - Resultados}: Exibe os principais resultados do projeto, com destaque para os diagramas UML elaborados. São apresentados o diagrama de casos de uso, o diagrama de sequência e o diagrama de classes, que evidenciam as funcionalidades, a arquitetura e as interações do sistema desenvolvido;

\item \textbf{Capítulo \ref{chap:conc} - Conclusão}: Discute as conclusões alcançadas com o desenvolvimento do projeto, ressaltando os aprendizados técnicos e interpessoais adquiridos pela equipe. Aponta ainda as contribuições do trabalho, o cumprimento dos objetivos propostos e sugestões de melhorias ou futuras evoluções do sistema.

\end{itemize}

