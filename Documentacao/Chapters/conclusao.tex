\chapter{Conclusão}
\label{chap:conc}

O desenvolvimento deste projeto proporcionou uma experiência prática completa em Análise e Projeto de Sistemas, permitindo à equipe aplicar conhecimentos teóricos em um cenário real com um cliente ativo. A criação do site portfólio envolveu etapas fundamentais como levantamento e análise de requisitos, modelagem de processos, elaboração de diagramas UML e desenvolvimento técnico com tecnologias amplamente utilizadas no mercado.

Ao longo do projeto, foi possível vivenciar as etapas do ciclo de vida de um sistema, desde a concepção inicial até a entrega final do produto. O uso de uma abordagem híbrida, mesclando o modelo Waterfall com práticas do Scrum, contribuiu para uma organização eficaz e para o acompanhamento contínuo das demandas do cliente.

Além dos aprendizados técnicos, o projeto também reforçou a importância da comunicação, da escuta ativa e do trabalho colaborativo. A entrega de um site funcional, visualmente coerente e que atende às expectativas do cliente demonstra não apenas a competência técnica da equipe, mas também seu compromisso com a qualidade e com o resultado final.

Assim, conclui-se que o projeto atingiu seus objetivos e proporcionou um desenvolvimento acadêmico e profissional significativo para todos os envolvidos, servindo como base sólida para futuros desafios na área de tecnologia.

\section{Considerações finais}
\label{sec:consid}

A execução deste projeto permitiu compreender, na prática, como a análise e o projeto de sistemas se conectam com as necessidades reais de um cliente. Mais do que aplicar ferramentas e técnicas, o processo exigiu sensibilidade para interpretar expectativas, adaptar soluções e tomar decisões com base em critérios técnicos e humanos.

Além do aprendizado técnico, o trabalho destacou o valor da empatia no relacionamento com o cliente, da responsabilidade com prazos e da colaboração entre os membros da equipe. Cada etapa, do levantamento de requisitos à entrega final, representou uma oportunidade de crescimento individual e coletivo.

Outro ponto importante foi a capacidade de adaptação: lidar com mudanças, revisar planos e ajustar entregas de acordo com o feedback recebido tornou-se um diferencial essencial para alcançar o resultado esperado.

