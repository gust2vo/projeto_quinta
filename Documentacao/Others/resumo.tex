\begin{thesisresumo}
Este relatório apresenta o desenvolvimento de um projeto prático da disciplina de Análise e Projeto de Sistemas, cujo objetivo foi criar um site portfólio personalizado para um cliente real. A pesquisa foi conduzida com base em uma abordagem mista, combinando elementos do modelo Waterfall com práticas do framework Scrum, permitindo maior controle e flexibilidade durante o desenvolvimento. Inicialmente, foram levantados os requisitos do cliente por meio de reuniões e formulários, e posteriormente elaborados os diagramas UML (casos de uso, sequência e classes), definição da arquitetura da solução, modelagem dos processos e escolha das tecnologias (HTML, CSS, JavaScript e GitHub Pages). O projeto permitiu a aplicação prática dos conceitos estudados, além do desenvolvimento de habilidades interpessoais. Como contribuição, o trabalho entrega não apenas um produto funcional e esteticamente alinhado às preferências do cliente, mas também um exemplo de documentação completa para projetos de software, reforçando a importância da engenharia de requisitos e da validação contínua junto ao cliente.

\ \\

% use de três a cinco palavras-chave

\textbf{Palavras-chave}: Análise de Sistemas, Projeto de Sistemas, Portfólio Web, Desenvolvimento Front-end, Requisitos de Software.

\end{thesisresumo}
