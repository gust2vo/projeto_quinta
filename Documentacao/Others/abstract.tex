\begin{thesisabastract}
This report presents the development of a practical project for the Systems Analysis and Design course, aimed at creating a personalized portfolio website for a real client. The research was conducted using a hybrid approach, combining elements of the Waterfall model with practices from the Scrum framework, enabling greater control and flexibility throughout development. Initially, client requirements were gathered through meetings and forms, followed by the creation of UML diagrams (use cases, sequence, and class), definition of the solution architecture, process modeling, and selection of technologies (HTML, CSS, JavaScript, and GitHub Pages). The project enabled the practical application of course concepts, as well as the development of interpersonal skills. As a contribution, the work delivers not only a functional product aligned with the client's aesthetic preferences but also a complete example of software project documentation, reinforcing the importance of requirements engineering and continuous client validation. 

\ \\

% use de tr�s a cinco palavras-chave

\textbf{Keywords}: Systems Analysis, Systems Design, Web Portfolio, Front-End Development, Software Requirements.

\end{thesisabastract}
