\chapter{Conceito do projeto do portfólio}
\label{chap:fundteor}
Para o desenvolvimento do site portfólio, foram utilizados HTML, CSS, JavaScript e a plataforma GitHub Pages \cite{W3SchoolsTutorial,GitHubPages}, cada um com uma função específica para garantir um resultado funcional, estético e de fácil acesso.


O \textbf{HTML} (HyperText Markup Language) foi usado para estruturar o conteúdo do site, definindo os elementos principais como textos, imagens, títulos, listas e links. Essa estrutura é a base de qualquer página web, organizando as informações de forma hierárquica e semântica.

O \textbf{CSS} (Cascading Style Sheets) foi responsável pela parte visual do site, controlando o layout, cores, fontes, espaçamentos e responsividade. Com o CSS, foi possível criar uma identidade visual atraente e garantir que o site se adapte a diferentes tamanhos de tela, como em celulares e desktops.

O \textbf{JavaScript} foi utilizado para adicionar interatividade e dinamismo à página, como animações, efeitos visuais e manipulação do conteúdo em resposta às ações do usuário, tornando a navegação mais fluida e intuitiva.

Por fim, o \textbf{GitHub Pages} foi escolhido como plataforma de hospedagem gratuita para o site. Ele permite que o projeto seja publicado diretamente a partir de um repositório no GitHub, facilitando a gestão do código, o versionamento e o acesso público ao portfólio de forma simples e eficiente.

Essa combinação de tecnologias garantiu a criação de um site estável, moderno e acessível, alinhado às necessidades do cliente e às boas práticas de desenvolvimento web \cite{Pressman2016,Sommerville2011}.

Lista dos documentos
\begin{enumerate}
   \item diagrama de classe
   \item diagrama de casos de uso
   \item diagrama de sequência \cite{Larman2007}
\end{enumerate}

%conferir se precisa de requisitos do cliente
\section{Requisitos do cliente}
\label{sec:requisitos}

O cliente definiu certos requisitos quanto à aparência, estilo e conteúdo do site portfólio. Esses requisitos foram coletados durante as reuniões iniciais e serviram como base para o desenvolvimento do projeto. Abaixo estão listados os principais pontos:

\begin{itemize}
    \item \textbf{Cores preferidas}: Preto e roxo;
    \item \textbf{Estilo visual}: Minimalista, com foco na simplicidade e clareza;
    \item \textbf{Navegação preferida}: Scroll único (todas as informações em uma única página);
    \item \textbf{Foto pessoal}: Deseja incluir uma foto pessoal no site;
    \item \textbf{Toque desejado no portfólio}: Pessoal e descontraído, refletindo a personalidade do cliente;
    \item \textbf{Formato da descrição pessoal}: Lista com tópicos, facilitando a leitura;
    \item \textbf{Conteúdo a ser evitado}: Nenhuma restrição quanto ao conteúdo do site.
\end{itemize}

Além do design, também foram levantadas informações sobre o perfil técnico do cliente, que influenciaram na organização do conteúdo:

\begin{itemize}
    \item \textbf{Tecnologias com que já teve contato}: Desenvolvimento web em geral;
    \item \textbf{Estudos em andamento}: JavaScript (atualmente estudando) e começará Python em breve;
    \item \textbf{Áreas de interesse}: Deseja aprofundar conhecimentos em React e UI Design;
    \item \textbf{Cursos/Certificações}: Possui certificados em HTML e JavaScript;
    \item \textbf{Estilo de aprendizagem}: Prefere aprender na prática e assistindo vídeos.
\end{itemize}

Também foram coletadas informações pessoais para personalização do site:

\begin{itemize}
    \item \textbf{Hobbies}: Jogar bola, ler, jogar jogos de computador e programar sites;
    \item \textbf{Séries ou filmes favoritos}: Dark (série) e Esposa de Mentirinha (filme);
    \item \textbf{Estilo musical}: Pagode e sertanejo;
    \item \textbf{Citações/frases}: Não deseja incluir nenhuma frase ou citação específica no site.
\end{itemize}

\section{Requisitos Funcionais}
\label{sec:reqfuncionais}

Os requisitos funcionais definem o comportamento e as funcionalidades que o sistema deverá apresentar para atender às necessidades do cliente \cite{Pressman2016,Sommerville2011}.


\begin{itemize}
    \item \textbf{RF01 - Exibir informações do desenvolvedor}:\\
    O sistema deve apresentar o nome, foto e uma breve descrição sobre o desenvolvedor.

    \item \textbf{RF02 - Apresentar links para redes sociais}:\\
    O sistema deve exibir ícones com links para o perfil do LinkedIn e do GitHub.

    \item \textbf{RF03 - Redirecionar para o LinkedIn}:\\
    Ao clicar no ícone do LinkedIn, o sistema deve abrir o perfil do usuário em uma nova aba.

    \item \textbf{RF04 - Redirecionar para o GitHub}:\\
    Ao clicar no ícone do GitHub, o sistema deve abrir o perfil do usuário em uma nova aba.

    \item \textbf{RF05 - Exibir diplomas ao clique}:\\
    O sistema deve permitir que o usuário visualize os diplomas de HTML e JavaScript ao clicar nos respectivos links.

    \item \textbf{RF06 - Apresentar diplomas em modal}:\\
    O sistema deve exibir o diploma clicado em um modal centralizado na tela.

    \item \textbf{RF07 - Fechar modal}:\\
    O sistema deve permitir que o usuário feche o modal com um botão (ícone de "X").

    \item \textbf{RF08 - Apresentar seções informativas}:\\
    O sistema deve apresentar as seções "Sobre Mim", "Estilo" e "Tecnologias", com conteúdo textual formatado.
\end{itemize}

